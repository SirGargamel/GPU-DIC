\documentclass[a4paper,12pt]{article}
\usepackage[utf8x]{inputenc}
\usepackage[czech, english]{babel}
\selectlanguage{english}
\usepackage[FM, EN, bwtitles, noheader]{tul}
\usepackage{hyperref}
\usepackage{graphicx}
\usepackage{todonotes}
\presetkeys{todonotes}{inline}{}
\newcommand{\classname}[1]{\texttt{#1}}
\TULphone{+420\,485\,353\,030}
\TULmail{Petr.Jecmen@tul.cz}
\date{September 13, 2013}

\usepackage{listings}
\definecolor{gray}{rgb}{0.4,0.4,0.4}
\definecolor{darkblue}{rgb}{0.0,0.0,0.6}
\definecolor{cyan}{rgb}{0.0,0.6,0.6}
\lstset{
	basicstyle=\ttfamily,
	breaklines=true,
	keywordstyle=\color{darkblue},
	commentstyle=\color{gray},
    frame=trbl,
    rulecolor=\color{black!30},
    xrightmargin=7pt,
	columns=fullflexible,
	showstringspaces=false,
	language=Java
}
\lstdefinelanguage{XML}
{
	morestring=[b]",
	morestring=[s]{>}{<},
	morecomment=[s]{<?}{?>},
	identifierstyle=\color{darkblue},
	keywordstyle=\color{cyan},
	morekeywords={xmlns,version,type}
}

\begin{document}
\listoftodos
\newpage
\logo
\\\vspace{6pt}
\begin{center}
\large{\bfseries Manuál k softwaru DIC-Voucher}
\\\vspace{1pc}
\small{
Ing. Petr Ječmen
\\\vspace{1pc}
Fakulta mechatroniky, informatiky a mezioborových studií\\
Technická Univerzita v~Liberci\\
Studentská 2\\
461 17 Liberec}
\end{center}
\newpage
\section{Úvod}
\todo{úvod k čemu softare je a co umí}

\newpage
\section{Ovládání aplikace}
\subsection{Úvodní okno}
\todo{screenshot okna, čiselný popis prvků}
\subsection{Výběr vstupu}
\todo{video / obrázky / config / binary}
\subsection{Zadávání ROI}
\todo{jak se ovládá, jak se chová}
\subsection{Expertní nastavení}
\todo{odkaz na sekci s konfigurací úlohy}
\subsection{Spůštění výpočtu}
\todo{nic extra, pouze poznámka, že nenastavené parametry se nastaví na výchozí (uživatel se nemusí starat)}
\subsection{Uložení}
\todo{uložení configu / bin. otisku}
\subsection{Výsledky}
\todo{zobrazení výsledků a případné uložení}

\newpage
\section{Konfigurační soubor úlohy}
\todo{detailní popis možných položek v konfiguračnícm souboru}

\newpage
\section{Doporučená nastavení}
\todo{velikost facetu, parametr výpočtu prodloužení, velikost okna pro lokální dohledávání atd.}

% --- Zdrojove kody
\newpage
\todo{spravne zdrojove kody}
\section{Source code examples}
\subsection{Instance creation and messaging}\label{appMsg}
\begin{lstlisting}[title=Constant declaration]
final UUID ID = UUID.randomUUID();	// define ID for messaging
\end{lstlisting}
\begin{lstlisting}[title=Client side code]
public static void main(String[] args) {
  Client client = Client.Client.initNewClient();
  client.getListenerRegistrator().setIdListener(ID, new MessageHandler());
}

class MessageHandler implements Listener<Identifiable> {
  @Override
  public Object receiveData(Identifiable data) {
    if (data instanceof Message) {
      Message m = (Message) data;
      System.out.println(m.getData());	// print content to console
      return m.getHeader();	// send valid response
    } else {
      return "ERROR";	// report error
    }
  }
}
\end{lstlisting}

\subsection{Job Management}\label{appJm}
\begin{lstlisting}[title=Server side code]
public static void main(String[] args) {
  double start = -100.0, end = 100.0; // define range of values
  double step = 0.01;
  // divide the range in small pieces and create a Set<double[]>, where double[] defines start and end of each sub-range and step size
  for (double[] d : subranges) {	// submit all jobs
    s.getJobManager().submitJob(d);
  }
  s.getJobManager().waitForAllJobs();	// wait until all jobs are complete
  // use s.getJobManager().getAllJobs() to get all submitted jobs and find the best result
}

class DataStore implements DataStorage {
  @Override
  public Object requestData(Object o) {
    switch (o.toString()) {
      case "data":
        return dataArray;	// return data
      default:
        return "Illegal data request";
    }
  }
}
\end{lstlisting}

\end{document}